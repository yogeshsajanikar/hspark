% Created 2016-03-17 Thu 14:10
\documentclass[garamond]{article}
\usepackage[latin1]{inputenc}
\usepackage[T1]{fontenc}
\usepackage{fixltx2e}
\usepackage{graphicx}
\usepackage{grffile}
\usepackage{longtable}
\usepackage{wrapfig}
\usepackage{rotating}
\usepackage[normalem]{ulem}
\usepackage{amsmath}
\usepackage{textcomp}
\usepackage{amssymb}
\usepackage{capt-of}
\usepackage{hyperref}
\setcounter{secnumdepth}{2}
\author{Yogesh Sajanikar}
\date{March 17, 2016}
\title{Implementing Apache Spark in Haskell}
\hypersetup{
 pdfauthor={Yogesh Sajanikar},
 pdftitle={Implementing Apache Spark in Haskell},
 pdfkeywords={},
 pdfsubject={},
 pdfcreator={Emacs 24.5.1 (Org mode 8.3.2)}, 
 pdflang={English}}
\begin{document}

\maketitle
\setcounter{tocdepth}{2}
\tableofcontents


\section{Abstract}
\label{sec:orgheadline1}
We present \href{https://github.com/yogeshsajanikar/hspark}{hspark}, a Haskell library to create Apache Spark\footnote{\url{http://spark.apache.org/}}
like RDDs, and do map-reduce on multiple nodes. We have selected
\emph{cloud haskell}\footnote{Cloud Haskell} to 

\section{Overview}
\label{sec:orgheadline2}
\href{http://spark.apache.org/}{Apache Spark} is a very popular framework for fast cluster
computing. It is reported to give performance benefits\footnotemark[1]{} above 
\href{http://hadoop.apache.org/}{Hadoop}.  
\end{document}
